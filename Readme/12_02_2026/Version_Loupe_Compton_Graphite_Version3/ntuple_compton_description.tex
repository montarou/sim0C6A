\documentclass[11pt,a4paper]{article}
\usepackage[utf8]{inputenc}
\usepackage[T1]{fontenc}
\usepackage[french]{babel}
\usepackage{amsmath,amssymb}
\usepackage{booktabs}
\usepackage{geometry}
\usepackage{xcolor}
\usepackage{hyperref}
\usepackage{siunitx}
\usepackage{enumitem}

\geometry{margin=2.2cm}
\hypersetup{colorlinks=true, linkcolor=blue!60!black, urlcolor=blue!60!black}

\newcommand{\ntuple}{\texttt{compton\_cone\_events}}
\newcommand{\code}[1]{\texttt{#1}}

\title{\vspace{-1.5cm}\textbf{Ntuple \ntuple{}}\\[4pt]
\large Enregistrement des diffusions Compton individuelles\\
dans le c\^one graphite --- Simulation Geant4 Mini~X}
\author{}
\date{\today}

\begin{document}
\maketitle
\vspace{-0.8cm}

%======================================================================
\section{Objectif}
%======================================================================

Le ntuple \ntuple{} enregistre \emph{chaque} diffusion Compton subie par un
photon primaire (parentID\,=\,0) dans le volume logique
\code{logicConeCompton} (c\^one graphite, $z \in [1.9,\;16.95]$~mm).
Il permet de tracer~:
%
\begin{itemize}[nosep]
  \item la carte $(z,\,r)$ des interactions Compton dans le c\^one,
  \item l'histogramme 2D $\theta_\text{in}$ \emph{vs} $\theta_\text{out}$,
  \item l'angle de diffusion $\theta_s$ en fonction de l'\'energie incidente,
  \item la perte d'\'energie $\Delta E$ \emph{vs} $E_\text{in}$,
  \item la v\'erification de la formule Compton
        $\Delta E = \dfrac{E_\text{in}^2\,(1 - \cos\theta_s)}
                          {m_e c^2 + E_\text{in}\,(1-\cos\theta_s)}$.
\end{itemize}

%======================================================================
\section{Structure du ntuple}
%======================================================================

Le ntuple est cr\'e\'e dans \code{AnalysisManagerSetup.cc} et rempli dans
\code{SteppingAction.cc} au sein du bloc de d\'etection Compton
(\code{procDefined->GetProcessName() == "compt"} dans
\code{logicConeCompton}). Il contient 16~colonnes~:

\begin{table}[h!]
\centering
\small
\begin{tabular}{@{}clcl@{}}
\toprule
\textbf{Col.} & \textbf{Nom} & \textbf{Type} & \textbf{Description} \\
\midrule
 0 & \code{eventID}           & \code{int}    & Identifiant de l'\'ev\'enement \\
 1 & \code{trackID}           & \code{int}    & Identifiant du track (photon primaire) \\
 2 & \code{n\_compton\_seq}   & \code{int}    & $n$-i\`eme Compton de ce track dans le c\^one \\
\midrule
 3 & \code{ekin\_before\_keV} & \code{double} & $E_\text{in}$ : \'energie cin\'etique avant (keV) \\
 4 & \code{ekin\_after\_keV}  & \code{double} & $E_\text{out}$ : \'energie cin\'etique apr\`es (keV) \\
 5 & \code{delta\_ekin\_keV}  & \code{double} & $\Delta E = E_\text{in} - E_\text{out}$ (keV) \\
\midrule
 6 & \code{x\_mm}             & \code{double} & Position $x$ de l'interaction (mm) \\
 7 & \code{y\_mm}             & \code{double} & Position $y$ (mm) \\
 8 & \code{z\_mm}             & \code{double} & Position $z$ (mm) \\
 9 & \code{r\_mm}             & \code{double} & $r = \sqrt{x^2 + y^2}$ (mm) \\
\midrule
10 & \code{theta\_in\_deg}    & \code{double} & $\theta_\text{in}$ : angle polaire incident (${}^\circ$, r\'ef. $+Z$) \\
11 & \code{phi\_in\_deg}      & \code{double} & $\varphi_\text{in}$ : angle azimutal incident (${}^\circ$) \\
12 & \code{theta\_out\_deg}   & \code{double} & $\theta_\text{out}$ : angle polaire sortant (${}^\circ$) \\
13 & \code{phi\_out\_deg}     & \code{double} & $\varphi_\text{out}$ : angle azimutal sortant (${}^\circ$) \\
14 & \code{scatter\_angle\_deg} & \code{double} & $\theta_s = \arccos(\hat{d}_\text{in}\cdot\hat{d}_\text{out})$ (${}^\circ$) \\
15 & \code{cos\_scatter}      & \code{double} & $\cos\theta_s$ \\
\bottomrule
\end{tabular}
\caption{Colonnes du ntuple \ntuple{}.}
\label{tab:columns}
\end{table}

%======================================================================
\section{Notes d'impl\'ementation}
%======================================================================

\begin{itemize}[nosep]
  \item Les directions incidente et sortante sont lues directement sur les
        \code{G4StepPoint}~:\\[2pt]
        \code{prePoint->GetMomentumDirection()} ($\hat{d}_\text{in}$)\quad et\quad
        \code{postPoint->GetMomentumDirection()} ($\hat{d}_\text{out}$).
  \item Dans Geant4, lors d'une diffusion Compton le photon diffus\'e
        \emph{continue comme le m\^eme track} (m\^eme \code{trackID},
        \code{parentID}\,=\,0). Seul l'\'electron de recul est cr\'e\'e
        comme secondaire. Le champ \code{n\_compton\_seq} permet de
        distinguer les diffusions multiples d'un m\^eme photon.
  \item Le marquage \code{MyTrackInfo} existant (flag
        \code{fComptonInCone}, compteur \code{fNComptonInCone}, derni\`ere
        position/\'energie) est conserv\'e pour le ntuple
        \code{plane\_passages} au plan $z = 18$~mm.
\end{itemize}

%======================================================================
\section{Fichiers modifi\'es}
%======================================================================

\begin{table}[h!]
\centering
\small
\begin{tabular}{@{}ll@{}}
\toprule
\textbf{Fichier} & \textbf{Modification} \\
\midrule
\code{include/AnalysisManagerSetup.hh} & D\'eclaration de \code{GetComptonConeNtupleId()} \\
\code{src/AnalysisManagerSetup.cc}     & Cr\'eation du ntuple (16 colonnes) + getter \\
\code{src/SteppingAction.cc}           & Remplissage \`a chaque Compton dans \code{logicConeCompton} \\
\code{analyse\_compton\_cone.C}        & Macro ROOT d'analyse (9 figures) \\
\bottomrule
\end{tabular}
\end{table}

%======================================================================
\section{Exploitation avec ROOT}
%======================================================================

Apr\`es recompilation et ex\'ecution du run~:
\begin{verbatim}
  root -l -b -q 'analyse_compton_cone.C("output.root")'
\end{verbatim}

Exemples de commandes interactives~:
\begin{verbatim}
  // Carte (z, r) des Compton
  compton_cone_events->Draw("r_mm:z_mm","","COLZ");

  // theta_in vs theta_out
  compton_cone_events->Draw("theta_out_deg:theta_in_deg","","COLZ");

  // DeltaE vs E_incident
  compton_cone_events->Draw("delta_ekin_keV:ekin_before_keV","","COLZ");

  // Perte relative
  compton_cone_events->Draw(
    "100*delta_ekin_keV/ekin_before_keV:ekin_before_keV",
    "ekin_before_keV>0","COLZ");
\end{verbatim}

\vfill
\noindent\rule{\textwidth}{0.4pt}
{\footnotesize Volume de donn\'ees estim\'e : $\sim$\,10\,000 entr\'ees / 1\,M
d'\'ev\'enements primaires ($\sim$\,1--2~MB dans \code{output.root}).}

\end{document}
