\documentclass[a4paper,11pt]{article}
\usepackage[utf8]{inputenc}
\usepackage[T1]{fontenc}
\usepackage[french]{babel}
\usepackage[margin=1.0cm, landscape]{geometry}
\usepackage{tikz}
\usetikzlibrary{arrows.meta, calc, decorations.markings, backgrounds}
\usepackage{siunitx}
\sisetup{locale=FR, per-mode=symbol}
\usepackage{amsmath}
\usepackage{float}
\usepackage{caption}

\pagestyle{empty}

\begin{document}

\begin{figure}[H]
\centering
\begin{tikzpicture}[
  xscale=0.97,
  yscale=0.97,
]

% ---- Couleurs fidèles à la figure de référence ----
\definecolor{cInox}{RGB}{60,65,75}       % gris-bleu foncé pour inox
\definecolor{cInoxFill}{RGB}{140,150,165} % remplissage inox
\definecolor{cGraphite}{RGB}{65,80,105}   % bleu-gris foncé graphite
\definecolor{cGraphFill}{RGB}{90,105,135} % remplissage graphite
\definecolor{cChannel}{RGB}{255,255,230}  % jaune très pâle (canal)
\definecolor{cBg}{RGB}{245,248,252}       % fond
\definecolor{cGreen}{RGB}{0,140,50}       % vert — transmission directe
\definecolor{cRed}{RGB}{210,60,60}        % rouge — γ incident
\definecolor{cBlue}{RGB}{50,120,210}      % bleu — Compton redirigé
\definecolor{cCyan}{RGB}{0,210,220}       % cyan — annotations cotes
\definecolor{cScore}{RGB}{210,35,35}      % rouge ScorePlane

% ==================================================================
%  FOND
% ==================================================================
\fill[cBg] (-1.5,-7.5) rectangle (22.5, 7.5);

% ==================================================================
%  GRILLE LÉGÈRE (optionnel, comme fond matplotlib)
% ==================================================================
\foreach \z in {0,5,10,15,20} {
  \draw[gray!15, thin] (\z,-7) -- (\z,7);
}
\foreach \r in {-6,-4,-2,0,2,4,6} {
  \draw[gray!15, thin] (-1,\r) -- (22,\r);
}

% ==================================================================
%  INOX 304 — blocs extérieurs (porte-collimateur)
% ==================================================================
% Bloc supérieur : z=0.5..17, r=3.17..6.5
\fill[cInoxFill]
  (0.5, 3.17) rectangle (17, 6.5);
\draw[cInox, thick]
  (0.5, 3.17) -- (17, 3.17) -- (17, 6.5) -- (0.5, 6.5) -- cycle;

% Bloc inférieur
\fill[cInoxFill]
  (0.5,-6.5) rectangle (17,-3.17);
\draw[cInox, thick]
  (0.5,-6.5) -- (17,-6.5) -- (17,-3.17) -- (0.5,-3.17) -- cycle;

% Labels inox
\node[font=\normalsize\bfseries, text=white, opacity=0.85]
  at (8.5, 5.0) {Inox 304};
\node[font=\small, text=white, opacity=0.7]
  at (8.5, 4.2) {(inchang\'e)};
\node[font=\normalsize\bfseries, text=white, opacity=0.85]
  at (8.5,-5.0) {Inox 304};
\node[font=\small, text=white, opacity=0.7]
  at (8.5,-4.2) {(inchang\'e)};

% ==================================================================
%  GRAPHITE — cône concentrateur
% ==================================================================
% Géométrie du cône :
%   Face source  (z=1.90) : R_in = 3.00, R_out = 3.15
%   Face sortie  (z=16.95): R_in = 1.00, R_out = 3.15
% En coupe (z,r), le graphite est un trapèze

% Moitié supérieure (r > 0)
\fill[cGraphFill]
  (1.90, 3.15) -- (16.95, 3.15) -- (16.95, 1.00) -- (1.90, 3.00) -- cycle;
\draw[cGraphite, semithick]
  (1.90, 3.15) -- (16.95, 3.15) -- (16.95, 1.00) -- (1.90, 3.00) -- cycle;

% Moitié inférieure (r < 0)
\fill[cGraphFill]
  (1.90,-3.15) -- (16.95,-3.15) -- (16.95,-1.00) -- (1.90,-3.00) -- cycle;
\draw[cGraphite, semithick]
  (1.90,-3.15) -- (16.95,-3.15) -- (16.95,-1.00) -- (1.90,-3.00) -- cycle;

% Labels graphite
\node[font=\large\bfseries, text=white]
  at (8.5, 2.3) {GRAPHITE};
\node[font=\normalsize, text=white]
  at (8.5, 1.6) {($\rho=2.26$)};

% ==================================================================
%  CANAL CENTRAL — rempli en jaune pâle
% ==================================================================
\fill[cChannel, opacity=0.5]
  (0, 0) -- (1.90, 3.00) -- (16.95, 1.00)
  -- (22, 1.00*22/16.95) -- (22, -1.00*22/16.95)
  -- (16.95,-1.00) -- (1.90,-3.00) -- cycle;
% Petite zone avant le cône
\fill[cChannel, opacity=0.3]
  (0,-3.17) rectangle (1.90, 3.17);

% ==================================================================
%  TRAJECTOIRES — TRANSMISSION DIRECTE (vert, θ < 3.2°)
% ==================================================================
% Angles directs : passent à travers le canal sans toucher le cône
% Max angle pour passer : atan(1.0/16.95) ≈ 3.38°
% À z=18, r = 18*tan(θ)

% Rayons positifs
\foreach \angle in {0.3, 0.7, 1.2, 1.7, 2.2, 2.7, 3.1} {
  \pgfmathsetmacro{\rend}{18*tan(\angle)}
  \draw[cGreen, thick, opacity=0.85]
    (0,0) -- (18, \rend);
}
% Rayons négatifs (symétrique)
\foreach \angle in {-0.3, -0.7, -1.2, -1.7, -2.2, -2.7, -3.1} {
  \pgfmathsetmacro{\rend}{18*tan(\angle)}
  \draw[cGreen, thick, opacity=0.85]
    (0,0) -- (18, \rend);
}
% Rayon axial
\draw[cGreen, thick] (0,0) -- (18, 0);

% ==================================================================
%  TRAJECTOIRES — γ INCIDENT SUR CÔNE (rouge/rose)
%  puis γ COMPTON REDIRIGÉ (bleu)
% ==================================================================
% Surface intérieure du cône : r(z) = 3.0 - (2.0/15.05)*(z - 1.9)
% Rayon depuis (0,0) à angle θ : r = z*tan(θ)
% Intersection : z_hit = 3.2525 / (tan(θ) + 0.13289)

% --- Moitié supérieure (r > 0) ---
% Hit points computed:
%   θ=7°   → z=12.72, r=1.56
%   θ=10°  → z=10.52, r=1.85
%   θ=13°  → z=8.58,  r=1.98
%   θ=17°  → z=7.01,  r=2.14
%   θ=22°  → z=5.72,  r=2.31
%   θ=28°  → z=4.73,  r=2.52
%   θ=35°  → z=3.90,  r=2.73
%   θ=45°  → z=3.04,  r=3.04 (near entrance, likely absorbed)

% Hit point 1: θ≈7°, z=12.72, r=1.56
\draw[cRed, thin, opacity=0.7] (0,0) -- (12.72, 1.56);
\fill[cBlue] (12.72, 1.56) circle (0.08);
\draw[cBlue, semithick, opacity=0.8] (12.72, 1.56) -- (18, 1.0);

% Hit point 2: θ≈10°, z=10.52, r=1.85
\draw[cRed, thin, opacity=0.7] (0,0) -- (10.52, 1.85);
\fill[cBlue] (10.52, 1.85) circle (0.08);
\draw[cBlue, semithick, opacity=0.8] (10.52, 1.85) -- (18, 0.9);

% Hit point 3: θ≈14°, z=8.20, r=2.05
\draw[cRed, thin, opacity=0.7] (0,0) -- (8.20, 2.05);
\fill[cBlue] (8.20, 2.05) circle (0.08);
\draw[cBlue, semithick, opacity=0.8] (8.20, 2.05) -- (18, 0.7);

% Hit point 4: θ≈18°, z=6.75, r=2.19
\draw[cRed, thin, opacity=0.7] (0,0) -- (6.75, 2.19);
\fill[cBlue] (6.75, 2.19) circle (0.08);
\draw[cBlue, semithick, opacity=0.8] (6.75, 2.19) -- (18, 0.5);

% Hit point 5: θ≈23°, z=5.55, r=2.36
\draw[cRed, thin, opacity=0.7] (0,0) -- (5.55, 2.36);
\fill[cBlue] (5.55, 2.36) circle (0.08);
\draw[cBlue, semithick, opacity=0.8] (5.55, 2.36) -- (18, 0.3);

% Hit point 6: θ≈30°, z=4.58, r=2.64
\draw[cRed, thin, opacity=0.7] (0,0) -- (4.58, 2.64);
\fill[cBlue] (4.58, 2.64) circle (0.08);
\draw[cBlue, semithick, opacity=0.8] (4.58, 2.64) -- (18, 0.1);

% Absorbed in graphite (red X): θ≈40° → near entrance
\draw[cRed, thin, opacity=0.7] (0,0) -- (3.20, 2.68);
\node[cRed, font=\large\bfseries] at (3.20, 2.68) {$\times$};

% Absorbed (red X) near z=2.5 — photon absorbed deep in cone
\draw[cRed, thin, opacity=0.5] (0,0) -- (2.3, 2.1);
\node[cRed, font=\large\bfseries] at (2.3, 2.1) {$\times$};

% --- Moitié inférieure (r < 0) ---
% Hit point 1: z=10.52, r=-1.85
\draw[cRed, thin, opacity=0.7] (0,0) -- (10.52,-1.85);
\fill[cBlue] (10.52,-1.85) circle (0.08);
\draw[cBlue, semithick, opacity=0.8] (10.52,-1.85) -- (18,-0.8);

% Hit point 2: z=7.5, r=-2.10
\draw[cRed, thin, opacity=0.7] (0,0) -- (7.5,-2.10);
\fill[cBlue] (7.5,-2.10) circle (0.08);
\draw[cBlue, semithick, opacity=0.8] (7.5,-2.10) -- (18,-0.4);

% Hit point 3: z=5.2, r=-2.40
\draw[cRed, thin, opacity=0.7] (0,0) -- (5.2,-2.40);
\fill[cBlue] (5.2,-2.40) circle (0.08);
\draw[cBlue, semithick, opacity=0.8] (5.2,-2.40) -- (18,-0.2);

% Hit point 4: z=4.3, r=-2.55
\draw[cRed, thin, opacity=0.7] (0,0) -- (4.3,-2.55);
\fill[cBlue] (4.3,-2.55) circle (0.08);
\draw[cBlue, semithick, opacity=0.8] (4.3,-2.55) -- (18, 0.1);

% Absorbed in cone (lower)
\draw[cRed, thin, opacity=0.5] (0,0) -- (3.5,-2.90);
\node[cRed, font=\large\bfseries] at (3.5,-2.90) {$\times$};

% ==================================================================
%  SOURCE γ — étoile rouge à l'origine
% ==================================================================
\node[cRed, font=\huge] at (0,0) {$\bigstar$};
\node[cRed, font=\normalsize\bfseries, anchor=east] at (-0.5, 0.6) {Source $\gamma$};

% ==================================================================
%  SCOREPLANE — z = 18 mm (ligne tiretée rouge)
% ==================================================================
\draw[cScore, very thick, dashed] (18,-7) -- (18, 7);
\node[cScore, font=\small\bfseries, rotate=90, anchor=south]
  at (18.3, 4) {ScorePlane};

% ==================================================================
%  ANNOTATIONS DIMENSIONNELLES (cyan)
% ==================================================================

% R = 3.0 mm (entrée) — flèche verticale
\draw[cCyan, thick, {Stealth}-{Stealth}]
  (1.90, 0.3) -- (1.90, 3.00);
\node[cCyan, font=\small\bfseries, anchor=east] at (1.6, 1.6) {3.0};
\node[cCyan, font=\small\bfseries, anchor=east] at (1.6, 1.0) {mm};
% Label "R=3.0mm (entrée)"
\node[cCyan, font=\scriptsize, anchor=north west, text opacity=0.9]
  at (2.1, 3.2) {$R\!=\!3.0$\,mm};
\node[cCyan, font=\scriptsize, anchor=north west, text opacity=0.9]
  at (2.1, 2.7) {(entr\'ee)};

% R = 1.0 mm (sortie) — flèche verticale
\draw[cCyan, thick, {Stealth}-{Stealth}]
  (16.95, -0.3) -- (16.95, 1.00);
\node[cCyan, font=\small\bfseries, anchor=west] at (17.15, 0.35) {1.0 mm};
% Label
\node[cCyan, font=\scriptsize, anchor=south west, text opacity=0.9]
  at (15.3, 1.15) {$R\!=\!1.0$\,mm};
\node[cCyan, font=\scriptsize, anchor=north west, text opacity=0.9]
  at (15.3, 1.0) {(sortie)};

% ==================================================================
%  LÉGENDE (boîte blanche, coin supérieur droit)
% ==================================================================
\begin{scope}[shift={(14.5, 5.8)}]
  \fill[white, opacity=0.92, rounded corners=2pt]
    (-0.3,-3.8) rectangle (7.8, 1.5);
  \draw[gray!60, thin, rounded corners=2pt]
    (-0.3,-3.8) rectangle (7.8, 1.5);

  % Ligne verte — transmis direct
  \draw[cGreen, thick] (0, 1.0) -- (1.2, 1.0);
  \node[font=\small, anchor=west] at (1.5, 1.0)
    {Transmis direct ($\theta < 3.2°$)};

  % Ligne rouge — γ incident
  \draw[cRed, thin] (0, 0.3) -- (1.2, 0.3);
  \node[font=\small, anchor=west] at (1.5, 0.3)
    {$\gamma$ incident sur c\^one};

  % Ligne bleue — Compton redirigé
  \draw[cBlue, semithick] (0,-0.4) -- (1.2,-0.4);
  \node[font=\small, anchor=west] at (1.5,-0.4)
    {$\gamma$ Compton redirig\'e $\to$ avant};

  % Point bleu — interaction Compton
  \fill[cBlue] (0.6,-1.1) circle (0.08);
  \node[font=\small, anchor=west] at (1.5,-1.1)
    {Interaction Compton};

  % Croix rouge — absorbé
  \node[cRed, font=\large\bfseries] at (0.6,-1.8) {$\times$};
  \node[font=\small, anchor=west] at (1.5,-1.8)
    {Absorb\'e (photo)};

  % Étoile source
  \node[cRed, font=\normalsize] at (0.6,-2.5) {$\bigstar$};
  \node[font=\small, anchor=west] at (1.5,-2.5)
    {Source ($z = 0$)};

  % ScorePlane
  \draw[cScore, thick, dashed] (0,-3.2) -- (1.2,-3.2);
  \node[font=\small, anchor=west] at (1.5,-3.2)
    {ScorePlane ($z = 18$\,mm)};
\end{scope}

% ==================================================================
%  BOÎTE PRINCIPE PHYSIQUE (jaune pâle, coin inférieur droit)
% ==================================================================
\begin{scope}[shift={(14.5, -4.0)}]
  \fill[yellow!15, rounded corners=2pt]
    (-0.3,-2.7) rectangle (7.8, 0.5);
  \draw[cRed, thin, rounded corners=2pt]
    (-0.3,-2.7) rectangle (7.8, 0.5);

  \node[font=\small\bfseries, text=cRed, anchor=north west]
    at (0, 0.3) {Principe :};
  \node[font=\small, anchor=north west, text width=7.2cm]
    at (0,-0.15)
    {$\gamma$ \`a grand angle $\to$ Compton
     dans graphite $\to$ redirig\'e
     vers l'avant avec
     $\Delta E/E < 4\,\%$ \`a \SI{10}{\kilo\electronvolt}};
\end{scope}

% ==================================================================
%  AXES
% ==================================================================
\draw[->, >=Stealth, thick] (-1,0) -- (22, 0)
  node[below right, font=\normalsize\bfseries] {$z$ (mm)};
\draw[->, >=Stealth, thick] (0,-7) -- (0, 7.3)
  node[above left, font=\normalsize\bfseries] {$r$ (mm)};

% Graduations z
\foreach \z in {0, 5, 10, 15, 20} {
  \draw[thick] (\z,-0.12) -- (\z, 0.12);
  \node[below, font=\small] at (\z,-0.25) {\z};
}

% Graduations r
\foreach \r in {-6,-4,-2, 2, 4, 6} {
  \draw[thick] (-0.12,\r) -- (0.12,\r);
  \node[left, font=\small] at (-0.25,\r) {\r};
}

% ==================================================================
%  TITRE
% ==================================================================
\node[font=\Large\bfseries, anchor=south]
  at (10, 7.6)
  {Configuration propos\'ee --- C\^one concentrateur en graphite + Inox};

\end{tikzpicture}

\caption{Coupe $(z, r)$ du c\^one Compton en graphite + Inox~304.
Vert : transmission directe ($\theta < 3.2°$).
Rouge : $\gamma$ incident sur le c\^one.
Bleu : $\gamma$ redirig\'e vers l'avant apr\`es diffusion Compton
($\Delta E/E < 4\,\%$ \`a 10\,keV).
Croix : absorption photo\'electrique.}
\label{fig:cone_compton}
\end{figure}

\end{document}
